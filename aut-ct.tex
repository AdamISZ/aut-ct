\documentclass[10pt,a4paper]{article}
%\usepackage{fullpage}
\usepackage{fancyvrb}
%\usepackage[latin1]{inputenc}
\usepackage{amsmath}
\usepackage{amsfonts}
\usepackage{amssymb}
\usepackage{framed}
\usepackage{pstricks}    %for embedding pspicture.
\usepackage{graphicx}
\usepackage{hyperref}
% (1) choose a font that is available as T1
% for example:
\usepackage{lmodern}

% (2) specify encoding
\usepackage[T1]{fontenc}

% (3) load symbol definitions
%\usepackage{textcomp}
\usepackage{ifxetex,ifluatex}
\usepackage{fixltx2e} % provides \textsubscript
\ifnum 0\ifxetex 1\fi\ifluatex 1\fi=0 % if pdftex
  \usepackage[T1]{fontenc}
  \usepackage[utf8]{inputenc}
  \usepackage{textcomp} % provides euro and other symbols
\else % if luatex or xelatex
  \usepackage{unicode-math}
  \defaultfontfeatures{Ligatures=TeX,Scale=MatchLowercase}
\fi
% needed for small caption fonts
%\usepackage[skip=2pt]{caption}

%\DeclareCaptionFormat{myformat}{\fontsize{8}{9}\selectfont#1#2#3}
%\captionsetup{format=myformat}
\providecommand{\tightlist}{%
  \setlength{\itemsep}{0pt}\setlength{\parskip}{0pt}}
\bibliographystyle{plain}

\setlength{\parindent}{0pt}
\begin{document}

\title{Anonymous usage tokens from Curve Trees}
\maketitle

The thinking behind the title will be explained only in Section 2. Initially, we cover specificially how a ZKP of knowledge of commitment opening, with a key image, works.

\section{Proof of Knowledge of Representation with Key Image}

(Caveat: there is nothing original here, this is just an application of standard $\Sigma-$ protocol techniques; \emph{however}, it needs to be checked that there is no error in this protocol, see also Section 1.4).

\vspace{0.3 in}

The term ``representation'' here is in agreement with what is seen in certain other places e.g. {[}\protect\hyperlink{anchor-2}{2}{]}, but it can also just be described as ``proof of knowledge of opening of a Pedersen commitment''.

\vspace{0.3 in}

Setting: a DLOG-hard, prime order, additively homomorphic group $(\mathbb{G}, \cdot)$ of order $n$. Additive notation is used.

Goal: a Prover wants to prove that, for commitments ($\in \mathbb{G}$) $C, C_2$, that they present, the following hold true:

\begin{itemize}
\item the Prover knows a secret witness $(x, \delta)$ such that the commitment $C$ is represented by those values with respect to generators $G, H$: $C = xG + \delta H$.
\item $C_2 = xJ$ for a third generator $J$.
\end{itemize}

(For the generators $G, H, J$ referred to, it's understood that they are elements of $\mathbb{G}$, and that their relative DLOGs are unknown).


\subsection{Setup}

\subsubsection{Setup for prover}
\begin{itemize}
\item Generate scalars $\in \mathbb{Z}_n$: $x, \delta$ - but note, the prover may already possess the $x$ such that it is the pre-image for a public key $P = xG$
\item Generate the commitment $C = xG + \delta H$
\item Generate the ``key image'' commitment $C_2 = xJ$
\item Agree on a context-specific message $m$.
\end{itemize}

\subsubsection{Setup for Verifier}
\begin{itemize}
\item Agree on a context-specific message $m$.
\item Receive from prover the claimed values $C, C_2$.
\end{itemize}


\subsection{Prover steps}

\begin{itemize}
\item Choose values $s, t \in \mathbb{Z}_n$
\item Calculate $R_1 = sG + tH$
\item Calculate $R_2 = sJ$
\item Calculate the Fiat-Shamir hash challenge: $e = \mathbb{H}(R_1, R_2, C, C_2, m)$
\item Calculate response 1: $\sigma_1 = s + ex$
\item Calculate response 2: $\sigma_2 = t + e\delta$
\item Send to Verifier: $\left(R_1, R_2, \sigma_1, \sigma_2\right)$
\end{itemize}

\subsection{Verifier steps}

Verifier receives the above message $\left(R, R_2, \sigma_1, \sigma_2\right)$.
\begin{itemize}
\item Calculate the Fiat-Shamir hash challenge: $e = \mathbb{H}(R_1, R_2, C, C_2, m)$
\item Check that $\sigma_1 G + \sigma_2 H \stackrel{?}{=} R_1 + eC$, reject if not
\item Check that $\sigma_1 J \stackrel{?}{=} R_2 + e C_2$, reject if not.
\item Accept.
\end{itemize}

\subsection{Security}
The above techniques are standard application of $\Sigma-$protocols. More specifically, this is a variant on the standard Schnorr-based DLEQ proof, where we prove knowledge of representation (a Pedersen commitment), instead of only the secret $x$, for the base generator $G$.

But some proof that the properties of \textbf{correctness, soundness and zero-knowledgeness} hold, should be presented here.

\section{In connection with ZK set membership}

The above protocol can be ``attached'' to the zero knowledge set membership protocol ``Curve Trees''{[}\protect\hyperlink{anchor-1}{1}{]}. In the notation of that paper, our $C$ here is $\hat{C}$, i.e the \textbf{rerandomized} value of an underlying curve element, which, in the EC group setting, can mean a public key $P$ such that its private key is the $x$ referred to in the above protocol.

By running the \textbf{Select-And-Rerandomize} algorithm on a curve tree over a set of such public keys ${P_i}$, again using the notation of that paper, and generating a proof $\Pi_{SR}$ for a prover-provided blinded value $\hat{C}$, connecting it to a root $rt$ of the tree, and then running the above algorithm to generate a ``key-image'' $C_2$ as per above, we can allow a verifier to know not only that the rerandomized public key was indeed in the tree, but that no further use of the same key is possible, without re-use of the same key image.

\textbf{Relation to Vcash}: It should be noted in this context, that the goal described here (the non-``double-spendability'' property for a commitment) is already achieved, as it must be, for the theoretical construct of ``Vcash'' in {[}\protect\hyperlink{anchor-1}{1}{]}, so that coins (which are really commitments) cannot be spent twice. But Vcash requires users to actively ``mint'' new coins, and then create proofs of valid spending. If we want to create trees non-interactively (i.e. without requesting actions from the owners of the ``coins'', in this case just public keys $P_i$, and then record their usage locally, this more basic ``key image'' idea might make more sense (in certain contexts).

To give the concrete example intended: imagine creating the Curve Tree from a large set of existing Bitcoin utxos (taproot specifically, so that the public keys are exposed without owner interaction; see the concept of ``spontaneous'' linkable ring signatures), it's necessary that tree construction is ``local'' to the prover and verifier separately, and that the update of key images can be local to the verifier (if they are managing access to their own resources only; for a more complex scenario where resource usage is being restricted across a distributed group, then some kind of global ledger or gossip network may be needed).

\section{References}
\begin{enumerate}
\def\labelenumi{\arabic{enumi}.}
\tightlist
\item
  \protect\hypertarget{anchor-1}{}{}``Curve Trees: Practical and Transparent Zero-Knowledge Accumulators'', Campanelli, Hall-Andersen, Kamp, 2022
  \url{https://eprint.iacr.org/2022/756}
 \item
  \protect\hypertarget{anchor-2}{}{}``A Graduate Course in Applied Cryptograhy'' v0.6, Boneh, Shoup, 2023
  \url{https://toc.cryptobook.us/book.pdf}
  \end{enumerate}
\end{document}